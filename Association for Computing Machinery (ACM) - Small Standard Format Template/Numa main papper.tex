\documentclass[acmsmall,natbib=false]{acmart}


%%
%% \BibTeX command to typeset BibTeX logo in the docs
\AtBeginDocument{%
  \providecommand\BibTeX{{%
    Bib\TeX}}}

%% Rights management information.  This information is sent to you
%% when you complete the rights form.  These commands have SAMPLE
%% values in them; it is your responsibility as an author to replace
%% the commands and values with those provided to you when you
%% complete the rights form.
\setcopyright{acmlicensed}
\copyrightyear{2018}
\acmYear{2018}
\acmDOI{XXXXXXX.XXXXXXX}


%%
%% These commands are for a JOURNAL article.
\acmJournal{JACM}
\acmVolume{37}
\acmNumber{4}
\acmArticle{111}
\acmMonth{8}

%%
%% Submission ID.
%% Use this when submitting an article to a sponsored event. You'll
%% receive a unique submission ID from the organizers
%% of the event, and this ID should be used as the parameter to this command.
%%\acmSubmissionID{123-A56-BU3}

%% Bibliography style
\RequirePackage[
  datamodel=acmdatamodel,
  style=acmauthoryear,
  ]{biblatex}

%% Declare bibliography sources (one \addbibresource command per source)
\addbibresource{software.bib}
\addbibresource{sample-base.bib}

%%
%% end of the preamble, start of the body of the document source.
\begin{document}

%%
{\title{Numa Stuff}}


\author{Kidus Workneh}
\affiliation{%
  \institution{University of Colorado Boulder}
  % \city{Hekla}
  \country{USA}}
\email{kidus.Workneh@colorado.edu}

\author{Pedro Kasprzykowski}
\affiliation{%
  \institution{University of Colorado Boulder}
  % \city{Hekla}
  \country{USA}}
\email{Pedro.Kasprzykowski@colorado.edu}

\author{Gowtham Kaki}
\affiliation{%
  \institution{University of Colorado Boulder}
  % \city{Hekla}
  \country{USA}}
\email{Gowtham.Kaki@colorado.edu}


\author{Joseph Izraelevitz}
\affiliation{%
  \institution{University of Colorado Boulder}
  % \city{Hekla}
  \country{USA}}
\email{Joseph.Izraelevitz@colorado.edu}

%%
%% The abstract is a short summary of the work to be presented in the
%% article.
\begin{abstract}
 \input{Abstract}
\end{abstract}

%%
%% The code below is generated by the tool at http://dl.acm.org/ccs.cfm.
%% Please copy and paste the code instead of the example below.
%%
% \begin{CCSXML}
% <ccs2012>
%  <concept>
%   <concept_id>00000000.0000000.0000000</concept_id>
%   <concept_desc>Do Not Use This Code, Generate the Correct Terms for Your Paper</concept_desc>
%   <concept_significance>500</concept_significance>
%  </concept>
%  <concept>
%   <concept_id>00000000.00000000.00000000</concept_id>
%   <concept_desc>Do Not Use This Code, Generate the Correct Terms for Your Paper</concept_desc>
%   <concept_significance>300</concept_significance>
%  </concept>
%  <concept>
%   <concept_id>00000000.00000000.00000000</concept_id>
%   <concept_desc>Do Not Use This Code, Generate the Correct Terms for Your Paper</concept_desc>
%   <concept_significance>100</concept_significance>
%  </concept>
%  <concept>
%   <concept_id>00000000.00000000.00000000</concept_id>
%   <concept_desc>Do Not Use This Code, Generate the Correct Terms for Your Paper</concept_desc>
%   <concept_significance>100</concept_significance>
%  </concept>
% </ccs2012>
% \end{CCSXML}

% \ccsdesc[500]{Do Not Use This Code~Generate the Correct Terms for Your Paper}
% \ccsdesc[300]{Do Not Use This Code~Generate the Correct Terms for Your Paper}
% \ccsdesc{Do Not Use This Code~Generate the Correct Terms for Your Paper}
% \ccsdesc[100]{Do Not Use This Code~Generate the Correct Terms for Your Paper}

% %%
% %% Keywords. The author(s) should pick words that accurately describe
% %% the work being presented. Separate the keywords with commas.

\keywords{Do, Not, Us, This, Code, Put, the, Correct, Terms, for,
  Your, Paper}

\received{20 February 2007}
\received[revised]{12 March 2009}
\received[accepted]{5 June 2009}

%%
%% This command processes the author and affiliation and title
%% information and builds the first part of the formatted document.
\maketitle

\section{Introduction}
\input{Intro}

\section{Background}
\input{background}

\section{Design}
\input{Design}

\section{Evaluation}
\input{Eval}

\section{Related Work}
\input{Related Work}


\end{document}
\endinput
%%
%% End of file `sample-acmsmall-biblatex.tex'.
